品川\cite{shinagawa}や中田\cite{nakata}の文献を参考に,各章の書き方をそれぞれの章で説明します.
本章では,論文全体の書き方および「はじめに」の書き方について説明します.

\section{論文執筆にあたって}
\subsection{本研究の全体像}
論文をいきなり書くのは大変です.論文を書き始める前に,
以下にしたがって研究の大まかなまとめを書いてみてください.

\subsubsection*{本研究のコンセプト:ポイント・主張したいことは何か?}
xxx

\subsubsection*{本研究の背景:本研究をなぜ研究するのか?}
xxx

\subsubsection*{本研究が取り扱う課題:本研究によって何を解決するのか?}
xxx

\subsubsection*{本研究の提案:どういう方法で課題を解決するのか?}
xxx

\subsubsection*{本研究の貢献点:先行・関連研究と比較して,何が違うのか?}
アドバイス:先行・関連研究をまとめて,表\ref{tab:novelty}のようにまとめて比較できると
本研究の立ち位置が整理がしやすいです.

\begin{table}[tb]
    \centering
    \begin{tabular}{c|cc}
        \hline \hline
        手法 & 観点A & 観点B \\
        \hline
        既存手法1(引用) & △ & × \\
        既存手法2(引用) & ◯ & × \\
        提案手法 & ◎ & ◎ \\
        \hline 
    \end{tabular}
    \caption{既存手法と本研究の提案手法の比較}
    \label{tab:novelty}
\end{table}

\subsection{論文の執筆順序}
論文を「はじめに」から順に書くのは大変です.
そのため,研究の全体像を把握した上で,以下の順序で論文を書くことをおすすめします.
無論,この順序に従う必要はありません.

「謝辞」はウォーミングアップとして,最初に書いてみましょう.
次に,「研究背景」や「関連研究」など,すでに自分の中で整理できている部分の執筆と並行して,
「実装」や「評価」など,まだ論文に書くに足らない部分の進捗を出しましょう.

\begin{enumerate}
    \item 謝辞
    \item 研究背景・関連研究
    \item 実装・評価
    \item 考察と今後の展望
    \item はじめに・おわりに
    \item 概要
\end{enumerate}

\subsection{論文の執筆にあたっての注意点}
\subsubsection{重要なこと}
\begin{itemize}
    \item \textbf{原稿のタイポや誤字脱字,凡ミスを普段からなるべく減らす努力をしましょう}.
    多いほど先生から本質的なコメントをもらいづらくなります.先生が使える時間は有限なので,
    些末なミスばかりだと深く読み込むことが難しく,少しずつ修正が進んで読める原稿になってきた段階になるころには締切が近づいており,
    突然ボコボコに修正する羽目になってお互い辛い目に遭います.
    こうならないためには,まず自分自身で,前述の書く順を守ること,Wordに文をコピペしてみてエラーをチェックすること,Cylint\cite{cylint}を用いてチェックすることを実施してみてください.
    その後,先輩や同期など学生に見てもらい,最後に先生に見てもらうという流れが良いと思います.
    \item \textbf{複数の人に見てもらいましょう}.
    一人にいっぱい見てもらうのは大変なので,「はじめに」の最初だけ,「概要」だけというように絞って意見をもらうのも手です.
    (ほぼ査読のレベルで頑張って読んでコメントをくれた人たちは謝辞に入れた方が良いので,名前をメモしておきましょう.)
    \item \textbf{バックアップをとりましょう}.
    なぜか修論時期に限ってマシンの故障が頻発します. 提出間近になってPCが壊れてしまい,
    提出できなくなるという悲惨な事故は起こり得ます. そうならないように,ローカル環境だけでなくOneDriveやGoogle Drive,
    GitHubなどのクラウドサービスにバックアップ体制を構築しておきましょう.
    バージョン管理が可能であるため,誤った更新やデータ損失が発生した場合に簡単に元の状態に戻せることや,
    GitHub の Issue や Pull Request,Permalink 機能を用いたレビューができることから,
    サイバーセキュリティ研究室では,GitHubの利用を推奨します\cite{cytex}.
    \item \textbf{時間を決めて毎日書きましょう}.
    論文を一度にがっと書くのは大変ですし,ペースもつかみにくいです.30分で時間を区切ったり,
    「今日は関連研究の一つ目を埋めよう」「図を描こう」など,
    短期的な目標を一つずつ設定すると執筆の困難さを低減することができます.
    (自分の文章自体をついでにチェックしたりもついついするので,タイポや誤字脱字の修正機会を増やせます)
\end{itemize}

\subsubsection{気をつけると良いこと}
\begin{itemize}
    \item 使う用語はなるべく統一しましょう.紛らわしいものは明確な線引きをする意識で書きましょう.同じ単語なのに,違う意味を持つことや,同じ意味を持つのに違う単語を使っていると,読み手が混乱します.
    \item 主語を省略しないようにしましょう.省略しすぎると,読み手がなにを指しているのか分からなくなります.そのほかにも論文では適さない表現がいくつかあります.見延\cite{minobe}がまとめているので,参考にしてください.
    \item 句読点「,.」(全角カンマピリオド)を使う時は注意してください.半角で入力されていると,後から一括変換しようにも思わぬ場所まで変換されてしまう羽目になります.
\end{itemize}

\subsubsection{原稿を見てもらったら}
論文の原稿添削で同じことを繰り返し指摘する側も大変ですが,指摘される側もまあ辟易とするかと思います.
今後同じ目に合わないように,復習をしましょう.おすすめは原稿ごとに指摘された内容をメモしておいて,
執筆が終わったらメモを見返して,ここの編集要求の意図は何だったのかを考えたり,どうすれば指摘を受けなくて済むのか考えてみるのがおススメです.
論文の書き方の本を片手に持ちながら,疑問を解消するのもぜひやってみてほしいと思います.
\textbf{執筆の終わった直後が,一番学習効率が良いです}.

\section{「はじめに」を書く際のアドバイス}
\begin{itemize}
    \item 書き出しを「近年」ではじめるのは,読み手によって近年が示す時間軸が異なるため禁止.何を実現する上で何が重要なのか,分野において何が重要なのかを本研究のコンセプトに紐づけて一言で書いてみましょう.
    \item 各段落の最初の一文だけをつなげておおよその流れが理解できるように書きましょう.つまり,まず各段落の最初の一文だけを書いてストーリーが通っているかを先生と一緒に確認し,OKが出たら各段落に詳細な内容を肉付けしていく,という手順で書いてみてください(これは「はじめに」に限らず,原稿全体で同様の手順で進めるのが理想的です.).
\end{itemize}